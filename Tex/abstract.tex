\section*{摘要}
为了实现服务灵活扩展,服务资源细化,服务快速升级等目的,软件体系结构转向了微服务结构,告别了传统的单体架构模式。当单体服务被拆分为多组微服务时,服务请求需要经过统一的微服务网关导向所需的资源微服务获得响应,这个过程涉及到服务调度的问题。
由于技术的异构性,服务之间存在非常复杂的调用关系,怎么解决服务之间的调度是一个亟待解决的问题。针对这一问题,我们提出了一种在服务网格下对服务进行智能调度的方法,来提供更加完善的微服务调度方式提高微服务系统的整体响应能力。本方法分为三个部分,第一部分是基于多维度指标综合的动态负载计算,从多个维度去对服务实例负载进行计算和分析服务实例真正状态;第二部分是考虑细粒度版本兼容性的路由策略,通过更加细粒度的路由策略,将服务流量分流到更加合适的实例中;第三部分是两阶段的服务请求路由机制,通过两种不同着重点的网关之间的相互配合来实现更加灵活的服务请求路由方式。
\newpage
\section*{摘要}
为了实现服务灵活扩展,服务资源细化,服务快速升级等目的,软件体系结构转向了微服务结构,告别了传统的单体架构模式。微服务架构将整体服务拆分为一组小服务,小服务之间通过轻量级协议进行通信。随着微服务体量的增大,服务编排问题日益凸显,为了解决这个问题提出了服务网格的概念,意在使微服务系统获得伸缩能力与局部故障处理能力,由于技术的异构性,服务之间存在非常复杂的调用关系,怎么解决服务之间的调度是一个亟待解决的问题。针对这一问题,我们提出了一种在服务网格下对服务进行智能调度的方法,来提供更加完善的微服务调度方式提高微服务系统的整体响应能力。本方法分为三个部分,第一部分是基于多维度指标综合的动态负载计算,从多个维度去对服务实例负载进行计算和分析服务实例真正状态;第二部分是考虑细粒度版本兼容性的路由策略,通过更加细粒度的路由策略,将服务流量分流到更加合适的实例中;第三部分是两阶段的服务请求路由机制,通过两种不同着重点的网关之间的相互配合来实现更加灵活的服务请求路由方式。
\newpage